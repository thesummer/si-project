\section{Introduction}
\label{sec:introduction}
%
The GIPER instrument consortium answer to the ESA \ac{AO}\cite{JUICE_AO} for the L1 class \ac{JUICE} mission.
%
%
\subsection{JUICE Mission Overview}
ESA L1 mission selected May 2012 in Cosmic Vision programme. Expected launch date 2022. 7.5 year cruise to Jupiter. Orbit insertion 2030 around Jupiter including phase studies of Europa and Callisto. September 2032 orbit insertion around Ganymede. Nominal mission end 2033. Russian Ganymede lander.
%
%
\section{Scientific Objectives}
%scientific investigation. Overall instrument capability, global mission goal, anticipated scientific performance on instrument, compared to similar instruments, discussion of synergies between different observations, list of assuptions for achieving science objectives: SC performance, orbit, other payloads, ground segment.
%
The scientific outcome of this instrument proposal is in accordance with ESA \ac{SciRD}\cite{SciRD} and addresses many of the scientific investigations proposed in the ESA JUICE Assessment Study Report\cite{yellowbook}.
%
%
\subsection{Introduction}
\subsection{Scientific Goals}
\subsection{Scientific Performance Requirements}
% 
%
\section{Instrument Performance}
%sensitivities, assumptions, critical analysis to environement parameters (radiation, degradation etc.), orbit positions.
%

\section{Technical Description and Design}
%
The proposed instrument has been designed in accordance to ESA \ac{EID-A} for the \ac{JUICE} mission\cite{EIDA}.
%
\subsection{Design Overview}
\subsection{Instrument Design Elements}
\subsection{Technical Resources}
\subsection{Instrument Spacecraft Requirements}
%thermal, power, mechanical mounting, EMC, especially radiation sensitive parts, radiation shielding and mitigation strategy, flight heritage of hardware, ressource budgets, 
This mission assumes that an altimeter instrument is included in the JUICE mission scientific instrument package. Altimeter is needed to estimate the surface clutter and surface slope.
%
The instrument consortia will provide an Instrument Operation Manual.
%
\section{Summary of Instrument Interfaces}
%data, mechanical, power, EMC, thermal,
%
\section{On-ground and In-flight Test and Calibration}
%test equipment requirements (i.e. thermal vacuum chamber, EMC labs etc.), EGSE, radar signal simulator(test of ground processing chain),
%
Functional, EMC, Thermal-Vacuum, Vibration.
%
%
For \ac{EGSE} a \ac{GIPER} raw signal simulator will be developed by the instrument consortia. The raw signal simulator will be similar to the one developed for the \ac{SHARAD} instrument\cite{Giovanni} and allow testing of the signal processing chain and to develop a Ganymede transfer function considering the orbit altitude and surface clutter and sub-surface dielectric interfaces. 
%
%
A set of calibration files and algorithms will be send to the JUICE \ac{SOC} for the raw to L1b data processing.
\begin{wrapfigure}{r}{0.3\textwidth}
\centering
\includegraphics[width=0.27\textwidth]{figures/MEGS}
\caption[caption]{Mars Echo Generation System used on the SHARAD instrument\cite{MEGS}}
\label{fig:MEGS}
\end{wrapfigure}
%
%
In-flight internal calibration (transmitted signal looped to receiver and data sent to ground)
In-flight external calibration (unprocessed data of reflected signals from a flat surface region of Ganymede is sent to ground)
%
%
\section{System Level Assembly, Integration and Verification}
Verification by test will be the main method of verification.

Vibration and thermal vacuum tests will be done at IRF in Kiruna. Shock tests may be performed at Chalmers University of Technology\cite{Jonsson}. 
%
\begin{table}[H]
\centering
\caption{Instrument testing for different instrument models}
\label{tab:instrument_testing}
\begin{tabular}{p{0.35\textwidth}p{0.10\textwidth}p{0.10\textwidth}p{0.10\textwidth}p{0.25\textwidth}}
\hline
\textbf{Test} & \textbf{STM} & \textbf{EM} & \textbf{PFM} & \textbf{Facility}\\
\hline
Mechanical Interface, Mass Inspection & - & - & - & \\
Electrical Performance & - & - & - & \\
Functional Test & - & - & - & \\
Strength Test & - & - & - & \\
Sine and Random Vibrations Test & - & - & - & \\
Shock Test & - & - & - & \\
Thermal Vacuum Test & - & - & - &\\
EMC Conducted and Radiated & - & - & - & \\
DC Magnetic Test & - & - & - &\\
\hline
\end{tabular}
\end{table}

%
\subsection{Requirements}
\subsection{Deliverable Models}
%
In appliance with EIDA-R005590\cite{EIDA}, three instrument models will be developed:
\begin{itemize}
\item \ac{STM} - For testing of the instrument structural and thermal interface to the spacecraft\\
\item \ac{EM} - To test and verify the instruments functional and technical requirements as well as the instrument performance. If required, this unit will be refurbished as an \ac{FS}.\\
\item \ac{PFM} - will be build using full flight standard components and tested for qualification and acceptance levels.
\end{itemize}
%
\subsection{System Level Testing}
%
\section{Flight Operations Concept}
%operational modes, calibrations, 
%
The instrument modes are inherited from the SHARAD instrument\cite{SHARAD_ppt}.
%
\subsection{Nominal Operations}
%
Operation modes: Low data rate, high data rate, calibration, receive only
%
\subsection{Other Modes}
%
Silent Modes: Off, Heating
Support modes: Check/init, standby, warm-up, idle
%
The instrument consortia will provide expert support to the JUICE \ac{MOC} and \ac{SOC} during the payload commissioning phase and at critical operations.
%
\section{Science Ground Segment Concept}
%operations requests, on-board software maintenance (test facilities to generate and test update)
%Further described in the Science Implementation Plan (SIP) - answer to SIRD
ESA ESTRACK will be used as ground station network. A JUICE \ac{MOC} established at ESOC and \ac{SOC} at ESAC.

\subsection{Implementation concept for the Science Ground Segment}
\subsection{Planning of payload operations}
During the ??? mission phases, the instrument consortia will submit science operations plans and perform maintenance and optimizations as required.
\subsection{On-Board Software Maintenance}
%
\section{Data Reduction, Scientific Analysis and Archival Plans}
It is expected that housekeeping data and raw science data will be sent from the JUICE \ac{MOC}, over internet, to the instrument team in LTU, Kiruna. To correctly remove surface clutter signals, it is required to simultaneously receive the data from the altimeter instrument. 
Quick-Look data analyser - to optimise efficiency and scientific return of instrument operations.
L1b raw data (un-calibrated science data) analyser
Data will be archived at LTU and also sent to JUICE science data archive.
%
\section{Organization}
%\subsection{Schedule}
\subsection{Management Structure}
(Please note, some of the contents in this section are fictive and should not be taken literally.)\\\\
%
\noindent
Jan Sommer is the instrument \ac{PI}. He has an extensive background studying planet geology, especially on Mars. This study will enhance our knowledge of planet inner structures, geology and provide better understanding of planet formations and evolution.\\\\
%
\noindent
Morten Olsen is the project manager. With experience as project manager for previous successful space instruments, he will manage the project schedules and budgets.\\\\
%
\noindent
Omair Sarwar is the technical manager. With extended engineering experience in radar systems, he will ensure that the instrument meets the performance requirements, proper instrument verification and qualification in accordance with ESA space standards.\\\\
%
\subsection{Budget}
ACME Space Agency is the \ac{LFA} for this instrument proposal. A \ac{LEO} has been issued ensuring funding for the project during the instrument development phase, in-flight operations and post operations activities.