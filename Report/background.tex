\section{Introduction}
\label{sec:introduction}
%
The GIPER instrument consortium answer to the ESA \ac{AO}\cite{JUICE_AO} for the L1 class \ac{JUICE} mission.
%
%
\subsection{JUICE Mission Overview}
ESA L1 mission selected May 2012 in Cosmic Vision programme. Expected launch date 2022. 7.5 year cruise to Jupiter. Orbit insertion 2030 around Jupiter including phase studies of Europa and Callisto. September 2032 orbit insertion around Ganymede. Nominal mission end 2033. Russian Ganymede lander.
%
%
\section{Scientific Objectives}
%
%
\subsection{Introduction\label{sub:Introduction-science}}
%
The JUICE mission will investigate
the Jupiter system with special interest in the moons Ganymede and
Europa. Most of the information about Jupiter and its satellites
originates from the flybys of Voyager 1 and 2 in the end of the 1970's and to the most extend from the Galileo mission started in 1989 and ended in 2003. The objective of the Galileo mission was to investigate the Jupiter system and especially the weather on Jupiter. Therefore most instruments were focusing on imaging and measuring of particles and fields. The JUICE mission will focus on getting an insight on the Galilean moons of Jupiter, especially Ganymede and Europa. The moons consist mainly of solid constituents, therefore a different set of instruments are needed for thorough investigation. The \ac{GIPER} instrument will mainly focus Ganymede, but since Europa and to some extend Callisto both also have a surface consisting mostly of ice,
investigations of these moons are also relevant\cite{pater2010planetary}.
Ganymede is the largest moon in the Jovian System and one of the four
Galilean moons. It was discovered in 1610 by Galileo Galilei. With
a mean radius of $2,634$~km Ganymede is the largest moon in the Solar
System and even larger than the planet Mercury. It travels around
Jupiter in an orbit with a semi-major axis of $1,070,400$~km and an
eccentricity of $0.00013$. It is therefore the third of the Galilean
moons\cite{pater2010planetary}. 

Two sub-surface sounders have flown to the planet Mars: \ac{MARSIS} on Mars Express and \ac{SHARAD} on Mars Reconnaissance Orbiter. Their main purpose is to find and map sub-surface
water reservoirs\cite{Mouginot2010}, but also new findings about
the surface evolution of Mars have been published\cite{Watters2006}.
The ice rich surface of the Jovian moons will enable \ac{GIPER} to penetrate
the surface much deeper than what was possible on Mars with its rocky
crust. Therefore, even more insight in the sub-surface structure of
Ganymede is expected to result from the mission. 
%
%
\subsubsection{Ganymede Surface Composition }
%
It is believed that Ganymede consists mainly of three layers which are
fully differentiated. The core consists of a hot iron alloy responsible for generating the intrinsic magnetic field. The second
layer is made of heavier rocky material and the third layer consists
mainly of water ice. The icy surface is expected to be around $800$~km
thick. On the boundary between the icy and the rocky layer large oceans
of liquid water may be present\cite{bagenal2007jupiter}.

The main understanding of the surface composition and the underlying
processes of the Ganymede formation originates from the Galileo mission
and partly from flybys of Voyager 1 and 2. A true color image
of the Ganymede surface taken from the Galileo orbiter is show in
figure \ref{fig:Ganymed-true-color}. It consists mainly of water
ice and is basically separated into regions with a low albedo, often
referred to as dark terrain, which covers about 34~\% of the surface
and regions of high albedo, named bright terrain respectively, which
covers 66~\%\cite{bagenal2007jupiter}. The boundaries between dark
and bright terrain are mostly sharp and distinct. The dark terrain
is covered with a much higher amount of impact craters. It is therefore
believed to be much older than the bright terrain which could be as
young as 1 billion years\cite{Showman2004}. Stereo images suggest
that the dark furrow terrain is generally higher than the grooved
bright terrain. This leads to the idea that bright terrain has probably
formed by tectonic processes of the (former) dark terrain together
with transport processes of snow or water to the surface like cryo-volcanism
(see also section \ref{sub:volcanism}) thereby smoothing the surface.
Impact craters of meteoroids therefore show up as bright white spots
on the white terrain\cite{bagenal2007jupiter,Patterson2010,Schenk2001,Showman1997,Showman2004}. 
%
\begin{figure}
\begin{centering}
\includegraphics[width=0.7\textwidth]{Figures/Ganymede_true_color}
\par\end{centering}

\caption{True color image of Ganymede's surface taken by the Galileo spacecraft.\label{fig:Ganymed-true-color}}
\end{figure}
%
%
%
\subsubsection{Formation of Surface and Sub-surface Processes\label{sub:volcanism}}
%
The processes causing the formation of the bright terrain from the
ancient dark terrain are believed to be tectonics and cryo-volcanism
although they are still not fully understood and part of active discussions\cite{Patterson2010,Schenk2001,Showman2004}.
Most models need a heat source strong enough to (partly) melt sub-surface
layers. The current orbital parameters of Ganymede would not be sufficient
to produce enough heat by tidal forces. Orbital calculations suggest
that Ganymede had a period where the eccentricity of its orbit reached
as high as 0.03\cite{Showman1997} which would cause enough tidal
heating to melt parts of the icy interior up to an extend where the
moon extends up to 1~\% causing major tectonic movements and the
formation of grabens\cite{Showman2004}. 

Another unknown problem is the transport of the then (partly) molten
ice or slush to the surface in order to fill graben. One proposition
is that icy ``volcanoes'' ejected low-viscous liquid water which
then flooded the graben before freezing. So far no strong evidences
for this volcanism like ejection centers or downstream patterns on
the horsts have been found\cite{Patterson2010}. This may be because
even the high resolution images obtained by the Galileo and Voyager
missions do not have a high enough resolution, the areas for which
high resolution pictures are available do not have pronounced enough
evidences or they are just not existent\cite{Patterson2010,Schenk2001,Showman2004}.
Another problem is that water or slush have a higher density than
ice, thus if tidal heating would melt ice, the water or slush would
sink deeper instead of rising to the top.

An approach to solve this issue proposed by Showman and Mosqueira
would also explain why there is no ice on top on the horsts and the
lack of flooding tracks. The general idea is that tectonics resurfaced
the dark terrain to contain horsts and grabens. These apply different
pressure to the underlying terrain. These pressure gradients could
actually lead to the circumstance that material with a negative buoyancy
like water or slush could move upwards but only below the grabens.
When the graben are filled with ice, the process automatically stops
because the necessary pressure imbalance from the terrain disappears
hence no ice could reach the high horsts\cite{Showman2004}. 

In figure \ref{fig:gradients} an example calculation for the gradients
due to terrain imbalance is presented. The terrain was modeled as
a sine wave with 30~km wavelength and an amplitude of 1~km (2~km
peak-to-peak) as shown in the top graphic of figure \ref{fig:gradients}.
\begin{figure}
\begin{centering}
\includegraphics[width=0.7\textwidth]{Figures/gradients}
\par\end{centering}

\caption{Example calculation for \textbf{(a)} a sinusoidal terrain with an
amplitude of 1~km and a wavelength of 30~km. The \textbf{(b)} pressure
gradient from the tectonic imbalance can compensate the negative buoyancy
of water in lower surface areas up to 5~km \textbf{(c)}. Figure taken
from \cite{Showman2004}\label{fig:gradients}}


\end{figure}
 The vector plot in the middle shows the resulting pressure gradients.
As can be seen underneath the graben they are directed upwards but
decrease exponentially with depth. The lower vector plot shows the
resulting gradients when considering water with a negative buoyancy.
As can be seen water could only move up to the surface when it is
created in a depth below 5~km. Further calculations performed by
Showman, Mosqueira et al. estimate that depth from where water or
slush can rise to the surface ranges from 5~km to 10~km. A possible
problem with this model may be that is would need at least 1 million
years of constant water production in order to transport enough water
to the surface to fill grabens, but the graben could relax gravitationally
earlier and thus stop the upward flow to soon\cite{Showman2004}. 

In order to find the dominant processes for the resurfacing of Ganymede's
dark terrain it would be essential to acquire measurements from the
subsurface interior additionally to (new) surface pictures. Ganymede
can be seen as a prototype for an icy body. Therefore investigating
the tectonic processes of Ganymede and its surface evolution will
not only provide more information about the formation of Ganymede
but also of its siblings Europa and Callisto as well as icy bodies
in general \cite{bagenal2007jupiter}.


\subsection{Scientific Goals\label{sub:Scientific-Goals}}

The scientific outcome of this instrument proposal is in accordance
with ESA \ac{SciRD}\cite{SciRD} and follows the scientific goals
described in the ESA JUICE Assessment Study Report\cite{yellowbook}.
Based on the short scientific introduction of Ganymede from section
\ref{sub:Introduction-science} the following scientific goals can
be identified:
\begin{itemize}
\item Generate a global map of the interior below the surface of Ganymede
to get more insight about different layers, their composition and
their distribution.
\item Find evidences for the tectonic processes which created the horsts
and grabens of the grooved terrain
\item Find out how the dark and bright terrain did evolve over time.
\item Find evidences for or against different cryo-volcanism scenarios
\item Get a more detailed map of the surface terrain compared to stereoscopic
imaging
\item Possibility to find habitable zones 
\end{itemize}

\subsection{Scientific Performance Requirements}

In order to achieve the goals described in section \ref{sub:Scientific-Goals}
the instrument should be able to penetrate the surface to at least
5~km. Attenuation of radar waves in the lower MHz spectrum in ice
is quite small which is beneficial for a high penetration depth, but
although it is quite certain that the upper surface mainly consists
of water ice there might be significant amounts of rocky material
at some parts due to the many meteoroid impacts after the accretion
phase of Ganymede. Therefore an appropriate margin for the penetration
depths should be considered. 

The vertical resolution should be in the range of 10~m -- 35~m to
give the chance to resolve the position and offset of the identified
layers with high accuracy. A typical width for the groves of the terrain
is 10~km, thus the horizontal resolution should not exceed this value
in order to correlate different vertical layers to the surface terrain. 

As the goal is to create a map of the whole surface of Ganymede it
is expected that even after preprocessing and compression a large
amount of data will be collected. An appropriate downlink capacity
should be reserved for the mission.




\section{Technical Description and Design}
%
The proposed instrument has been designed in accordance to ESA \ac{EID-A} for the \ac{JUICE} mission\cite{EIDA}.
%
\subsection{Design Overview}
\subsection{Instrument Design Elements}
\subsection{Technical Resources}
\subsection{Instrument Spacecraft Requirements}
%thermal, power, mechanical mounting, EMC, especially radiation sensitive parts, radiation shielding and mitigation strategy, flight heritage of hardware, ressource budgets, 
This mission assumes that an altimeter instrument is included in the JUICE mission scientific instrument package. Altimeter is needed to estimate the surface clutter and surface slope.
%
\section{Summary of Instrument Interfaces}
%data, mechanical, power, EMC, thermal,
%
\section{On-ground and In-flight Test and Calibration}
%test equipment requirements (i.e. thermal vacuum chamber, EMC labs etc.), EGSE, radar signal simulator(test of ground processing chain),
%
Functional, EMC, Thermal-Vacuum, Vibration.
%
%
For \ac{EGSE} a \ac{GIPER} raw signal simulator will be developed by the instrument consortia. The raw signal simulator will be similar to the one developed for the \ac{SHARAD} instrument\cite{Giovanni} and allow testing of the signal processing chain and to develop a Ganymede transfer function considering the orbit altitude and surface clutter and sub-surface dielectric interfaces. 
%
%
\begin{wrapfigure}{r}{0.3\textwidth}
\centering
\includegraphics[width=0.27\textwidth]{Figures/MEGS}
\caption[caption]{Mars Echo Generation System used on the SHARAD instrument\cite{MEGS}}
\label{fig:MEGS}
\end{wrapfigure}
%
%
In-flight internal calibration (transmitted signal looped to receiver and data sent to ground)
In-flight external calibration (unprocessed data of reflected signals from a flat surface region of Ganymede is sent to ground)
%
%
\section{System Level Assembly, Integration and Verification}
%
%
\subsection{Requirements}
\subsection{Deliverable Models}
%
Two models of the proposed instrument will be developed:
\begin{itemize}
\item Engineering Model - To test and verify the instruments functional and technical requirements as well as the instrument performance.\\
\item Protoflight Model - will be build using full flight standard components and tested at qualification levels.
\end{itemize}
%
\subsection{System Level Testing}
%
\section{Flight Operations Concept}
%operational modes, calibrations, 
%
The instrument modes are inherited from the SHARAD instrument\cite{SHARAD_ppt}.
%
\subsection{Nominal Operations}
%
Operation modes: Low data rate, high data rate, calibration, receive only
%
\subsection{Other Modes}
%
Silent Modes: Off, Heating
Support modes: Check/init, standby, warm-up, idle
%
%
%
\section{Science Ground Segment Concept}
%operations requests, on-board software maintenance (test facilities to generate and test update)
%Further described in the Science Implementation Plan (SIP) - answer to SIRD
%
\section{Data Reduction, Scientific Analysis and Archival Plans}
%
\section{Organization}
%\subsection{Schedule}
\subsection{Management Structure}
(Please note, some of the contents in this section are fictive and should not be taken literally.)\\\\
%
\noindent
Dr. Jan Sommer is the instrument \ac{PI}. He has an extensive background studying planet geology, especially on Mars. This study will enhance our knowledge of planet inner structures, geology and provide better understanding of planet formations and evolution.\\\\
%
\noindent
Morten Olsen is the project manager. With experience as project manager for previous successful space instruments, he will manage the project schedules and budgets.\\\\
%
\noindent
Omair Sarwar is the technical manager. With extended engineering experience in radar systems, he will ensure that the instrument meets the performance requirements, proper instrument verification and qualification in accordance with ESA space standards.\\\\
%
\subsection{Budget}
ACME Space Agency is the \ac{LFA} for this instrument proposal. A \ac{LEO} has been issued ensuring funding for the project during the instrument development phase, in-flight operations and post operations activities.